\documentclass[12pt,oneside,slovak,a4paper]{article}

\usepackage[slovak]{babel}
\usepackage[utf8]{inputenc}
\usepackage{amsmath}
\usepackage{amsfonts}
\usepackage{amssymb}
\usepackage{graphicx}
\usepackage{cite}
\usepackage[IL2]{fontenc} % lepšia sadzba písmena Ľ než v T1
\usepackage{pdfpages}
\usepackage{url} % príkaz \url na formátovanie URL
\usepackage[hidelinks]{hyperref} % odkazy v texte budú aktívne (pri niektorých triedach dokumentov spôsobuje posun textu)
\usepackage[left=2cm,right=2cm,top=2cm,bottom=2cm]{geometry}
\usepackage{float}
\usepackage[normalem]{ulem}
\useunder{\uline}{\ul}{}
\usepackage{titling}
\usepackage{xcolor}
\usepackage{lipsum}
\usepackage{setspace}
\usepackage{blindtext}
\usepackage{caption}
\usepackage{tabularx}
\usepackage[numbers]{natbib}




% riadkovanie 1.5
\begin{document}
\linespread{1.5}\selectfont

\begin{titlepage}
	\centering
    {\Large Slovenská technická univerzita v Bratislave\par}
    {\Large Fakulta informatiky a informačných technológií\par}
	\vspace{7cm}
	{\huge\bfseries Spoľahlivý systém ukladania osobných dát v domácnosti\par}
	\vspace{0.5cm}
    {\Large \textsc{Projektovanie aplikácií počítačov}\par}
    \vspace{1cm}
	{\Large\itshape Marek Čederle\par}
    {\small\texttt{xcederlem@stuba.sk}\par}
	\vfill

	{\large \today\par}
\end{titlepage}


% ----------------- Obsah -----------------
\tableofcontents
\vspace*{\fill}
\newpage

% ----------------- Kapitoly -----------------

% vsetky zdroje z literatura.bib aby sa zobrazili bez citovania
\nocite{*}

\section{Úvod}
V dobe, kedy digitálne údaje sú veľmi dôležité nielen pre firmy, ale aj pre jednotlivcov, je nevyhnutné mať spoľahlivé riešenie na ukladanie a zálohovanie údajov. Tento projekt sa zameriava na implementáciu spoľahlivého systému ukladania a zálohovania údajov, ktorý môže bežná osoba prevádzkovať v pohodlí svojho domova. Pôjde o vytvorenie systému NAS\footnote{Network Attached Storage} využívajúcej bežne dostupné počítačové komponenty a sieťovú infraštruktúru. Projekt sa venuje aj výberu vhodného softvéru, ktorý efektívne využije možnosti systému NAS. Okrem toho sa zaoberá umiestnením NAS v domácom prostredí, aby bolo zabezpečené pohodlné používanie bez zbytočného rušenia. Dôležitou súčasťou projektu je aj porovnanie nákladov medzi domácou a komerčnou alternatívou.

\section{Hardvér}
Z dôvodu aby sme čo najviac ušetrili peniaze a pomohli planéte pomocou toho že sa budeme riadiť podľa hesla ``Reduce, Reuse, Recycle'' sme sa rozhodli že použijeme starý počítač, ktorý máme doma. Tento počítač bude slúžiť ako základ pre náš NAS systém na ktorom môžeme stavať ďalej alebo ho prípadne vylepšiť.

\subsection{Základná doska}

\subsection{Procesor}

\subsection{Pamäť}

\subsection{Disky}

\subsection{Zdroj}

\subsection{Sieťová karta}

\subsection{Záložný zdroj v podobe UPS}

\section{Softér}

\subsection{RAID}

\subsection{Operačný systém}
TrueNAS or open media vault
Setupnut SMB na sieti na pristup

\section{Umiestnenie v domácnosti}

\begin{figure}[H]
	\centering
	\captionsetup{justification=centering,margin=2cm}
	\includegraphics[width=\linewidth]{./images/nakres-bytu.png}
	\centering
	\caption{Pôdorys bytu \\ Zdroj: None, toto som tu nechal len tak}
\end{figure}

\section{Porovnanie s inými riešeniami}

\subsection{Komerčné riešenia}

\subsection{DIY riešenia}
Raspberry Pi + disky s OpenMediaVault
Video od LTT

\section{Záver}

% bulleted list plus riadkovanie iba na danu cas textu
\setstretch{1.0}
\begin{itemize}
	\item 
		\begin{itemize}
			\item 
		\end{itemize}
	\item 
		\begin{itemize}
			\item 
		\end{itemize}
	\item 
		\begin{itemize}
			\item 
		\end{itemize}
	\item 
		\begin{itemize}
			\item 
		\end{itemize}
	\item 
		\begin{itemize}
			\item 
		\end{itemize}
\end{itemize}


\bibliography{literatura}
\bibliographystyle{unsrtnat}
\end{document}