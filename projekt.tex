\documentclass[12pt,oneside,slovak,a4paper]{article}

\usepackage[slovak]{babel}
\usepackage[utf8]{inputenc}
\usepackage{amsmath}
\usepackage{amsfonts}
\usepackage{amssymb}
\usepackage{graphicx}
\usepackage{cite}
\usepackage[IL2]{fontenc} % lepšia sadzba písmena Ľ než v T1
\usepackage{pdfpages}
\usepackage{url} % príkaz \url na formátovanie URL
\usepackage{hyperref} % odkazy v texte budú aktívne (pri niektorých triedach dokumentov spôsobuje posun textu)
\usepackage[left=2cm,right=2cm,top=2cm,bottom=2cm]{geometry}
\usepackage{float}
\usepackage[normalem]{ulem}
\useunder{\uline}{\ul}{}
\usepackage{titling}


%\title{   
%Voľne šíriteľné nástroje na obnovu zmazaných súborov}
%\author{Marek Čederle\\[2pt]
%	{\small Slovenská technická univerzita v Bratislave}\\
%	{\small Fakulta informatiky a informačných technológií}\\
%	{\small \texttt{xcederlem@stuba.sk}}
%	}
%\date{\small \today}
%
\begin{document}

%\vspace{50pt}
%\maketitle
%\vspace*{\fill}
%\pagebreak

\begin{titlepage}
	\centering
	{\LARGE \textsc{}\par}
	\vspace{5cm}
	{\Large \textsc{Projektovanie aplikácií počítačov}\par}
	\vspace{1.5cm}
	{\huge\bfseries Úložiská dát serverových aplikácií\par}
    \vspace{1cm}
    {\large Slovenská technická univerzita v Bratislave\par}
    {\large Fakulta informatiky a informačných technológií\par}
	\vspace{1cm}
	{\Large\itshape Marek Čederle\par}
    {\small\texttt{xcederlem@stuba.sk}\par}
	\vfill

% Bottom of the page
	{\large \today\par}
\end{titlepage}


\tableofcontents
\vspace*{\fill}

\section{Úvod}
Tu môžete predstaviť\cite{TEST} svoju tému a poskytnúť stručný prehľad toho, o čom budete vo svojom projekte diskutovať.

\section{Lorem Ipsum}
V tejto časti môžete diskutovať o svojej analýze súborových systémov.

\section{donor epsum}
Tu môžete diskutovať o svojej analýze voľne dostupných nástrojov na obnovu dát. Môžete tiež spomenúť, prečo ste si pre svoj projekt vybrali konkrétny nástroj (napr. testdisk).

\section{dalsia sekcia}
V tejto časti môžete diskutovať o svojich experimentoch s vybraným nástrojom na obnovu údajov. Môžete poskytnúť podrobnosti o experimentoch, ktoré ste vykonali, získaných výsledkoch a svojich pozorovaniach.

\section{Záver}
Tu môžete zhrnúť svoje zistenia a poskytnúť záver pre váš projekt.

\bibliography{literature}
\bibliographystyle{alpha}
\end{document}