\documentclass[12pt,oneside,slovak,a4paper]{article}

\usepackage[slovak]{babel}
\usepackage[utf8]{inputenc}
\usepackage{amsmath}
\usepackage{amsfonts}
\usepackage{amssymb}
\usepackage{graphicx}
\usepackage{cite}
\usepackage[IL2]{fontenc} % lepšia sadzba písmena Ľ než v T1
\usepackage{pdfpages}
\usepackage{url} % príkaz \url na formátovanie URL
\usepackage[hidelinks]{hyperref} % odkazy v texte budú aktívne (pri niektorých triedach dokumentov spôsobuje posun textu)
\usepackage[left=2cm,right=2cm,top=2cm,bottom=2cm]{geometry}
\usepackage{float}
\usepackage[normalem]{ulem}
\useunder{\uline}{\ul}{}
\usepackage{titling}
\usepackage{xcolor}
\usepackage{lipsum}
\usepackage{setspace}
\usepackage{blindtext}
\usepackage{caption}
\usepackage{tabularx}
\usepackage[numbers]{natbib}




% riadkovanie 1.5
\begin{document}
\linespread{1.5}\selectfont

\begin{titlepage}
	\centering
    {\Large Slovenská technická univerzita v Bratislave\par}
    {\Large Fakulta informatiky a informačných technológií\par}
	\vspace{7cm}
	{\huge\bfseries Spoľahlivý systém ukladania osobných dát v domácnosti\par}
	\vspace{0.5cm}
    {\Large \textsc{Projektovanie aplikácií počítačov}\par}
    \vspace{1cm}
	{\Large\itshape Marek Čederle\par}
    {\small\texttt{xcederlem@stuba.sk}\par}
	\vfill

	{\large \today\par}
\end{titlepage}



% ----------------- Abstrakt -----------------

\newpage
\vspace*{\fill}
\begin{abstract}
    Cieľom môjho projektu bude zamerať sa, akým spôsobom si môže bežný človek sprevádzkovať spoľahlivý systém na ukladanie, prípadne zálohovanie dát. Bude sa jednať o návrh a sprevádzkovanie dátového servera (NAS) v domácnosti, ktorý sa bude skladať z bežných počítačových komponentov a jeho zapojenia v počítačovej sieti vrátane potrebnej kabeláže. Neoddeliteľnou súčasťou môjho projektu bude vhodný výber SW riešenia, ktoré bude na tomto HW ``bežať''. Taktiež pôjde o jeho umiestnenie v domácnosti vzhľadom na hlučnosť tohto systému. Na záver by som chcel porovnať cenu s konkurencieschopným komerčným riešením.
\end{abstract}
\vspace*{\fill}
\newpage

% ----------------- Obsah -----------------
\tableofcontents
\vspace*{\fill}

% ----------------- Kapitoly -----------------

% vsetky zdroje z literatura.bib aby sa zobrazili bez citovania
\nocite{*}

\section{Úvod}
Lorem pisum Lorem pisumLorem pisumLorem pisum.

\section{Lorem Ipsum}
Lorem pisumLorem pisumLorem pisum.

\section{donor epsum}
Lorem pisumLorem pisumLorem pisumLorem pisum.

\section{dalsia sekcia}
Lorem pisumLorem pisumLorem pisumLorem pisumLorem pisumLorem pisumLorem pisum.

\section{Záver}
Lorem pisumLorem pisumLorem pisumLorem pisumLorem pisum.

\bibliography{literatura}
\bibliographystyle{unsrtnat}
\end{document}